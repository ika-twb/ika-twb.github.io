% Created 2021-11-14 Sun 16:17
% Intended LaTeX compiler: xelatex
\documentclass[11pt]{article}
\usepackage{graphicx}
\usepackage{grffile}
\usepackage{longtable}
\usepackage{wrapfig}
\usepackage{rotating}
\usepackage[normalem]{ulem}
\usepackage{amsmath}
\usepackage{textcomp}
\usepackage{amssymb}
\usepackage{capt-of}
\usepackage{hyperref}
\date{\today}
\title{Elements of Set Theory : note}
\hypersetup{
 pdfauthor={},
 pdftitle={Elements of Set Theory : note},
 pdfkeywords={},
 pdfsubject={},
 pdfcreator={Emacs 27.2 (Org mode 9.4.4)}, 
 pdflang={English}}
\begin{document}

\maketitle
\tableofcontents

\#+AUTHOR Ilghar Kus

This is my note during the reading
of the book \emph{Elements of Set Theory} by \emph{Herbert B. Enderton} .

The book is great.
\section{Chapter 1 \emph{INTRODUCTION}}
\label{sec:org94df578}
\subsection{Baby Set Theory}
\label{sec:org8b0b681}
In a naive approach, here is some definition.
\begin{itemize}
\item A \emph{set} is a collection of things(called its \emph{members} or \emph{elements}),
and the collection being regarded as a singel object.
\item \(t\in A\) means \(t\) is a member of \(A\), and \(t\notin A\) means \(t\) is not a member of \(A\).
\item \(A==B\) could be translated that \(A\) has the exact same members as \(B\), in which case,
\(A\) could be \(\{2,3,5,7\}\) and \(B\) could be the set of all solutions to the equation
\(x^4-17x^3+101x^2-247x+210=0\)

to which, we say:
\end{itemize}

\textbf{Principle of Extensionality} \(\quad\) If two sets have exactly the same members, 
then they are equal.

In a less naive approach, we could say:

\textbf{Principle of Extensionality} \(\quad\) If \(A\) and \(B\) are sets s.t. for \(\forall t\)
\[
   t\in A \iff t\in B
   \]
then \(A=B\).

\begin{itemize}
\item Empty set, noted as \(\emptyset\), has no members at all.
and it's the only set with no members.

\item \(\emptyset \in \{\emptyset\}\) and \(\emptyset \notin \emptyset\), so \(\emptyset \neq \{\emptyset\}\)

\item \textbf{Union and Intersection} \(\quad\) noted respectively as \(\cup\) and \(\cap\),
means all elements of \(A\) and/or \(B\) and all elements of \(A\) and \(B\)

\item The set of all subsets of \(A\) is noted as \emph{power set} \(\mathscr{P} A\) of \(A\),
also could be noted as \(2^{A}\)

\item \emph{Method of abstraction} \(\quad\) is a very flexible way of naming a set.
Notation used for the set of all objects \(x\) s.t. the condition P(x) holds is
\[
     \{x\|P(x)\}
     \]
\end{itemize}
\end{document}